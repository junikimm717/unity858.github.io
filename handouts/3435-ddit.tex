\documentclass{seto}
\title{Desargues and his meme }
\author{Neal + Krishna (feat. Tiger)}
\date\today
\begin{document}
\maketitle
We are lazy, so here is a problem compilation, one of the shortest handouts of all time.
\begin{block}[Very brief blurb]
Desargues was a sinner, a mastermind spinner,\\
Mathematical genius, his thoughts got much bigger.\\
With lines and planes, he played his wicked game,\\
Geometry was his realm, and he conquered the terrain.\\[6pt]

He twisted and turned, in his mathematical maze,\\
Proving the theorems that left others amazed.\\
His mind was a canvas, where concepts would collide,\\
Creating new dimensions, in which truths would reside.\\[6pt]

From perspective, he derived his duality,\\
Projective geometry, his art with clarity.\\
He saw the world in a different light,\\
Unveiling hidden symmetries, day and night.\\[6pt]

Desargues danced with angels and demons alike,\\
Challenging the norms, never afraid to strike.\\
His sins were his passion, his rebellion was clear,\\
In a world of shapes and numbers, he had no fear.\\[6pt]

So raise a toast to Desargues, the sinner with a vision,\\
Whose mathematical legacy defies all derision.\\
For in his wickedness, he found the truth,\\
And left us with a geometric marvel, in our youth.\\
\hfill-- ChatGPT '23
\end{block}
\setcounter{section}{-1}
\section{Acknowledgement}
Eric Shen for teaching me this black magic and its very interesting applications. Thanks so much Eric! -- Neal
\section{Opening examples}
\begin{block}[Example 1 (OMMC Main 2023/24 by Tiger)]
Define acute $\triangle ABC$ with circumcenter $O$. The circumcircle of $\triangle ABO$ meets segment $BC$ at $D \ne B$, segment $AC$ at $F \ne A$, and the Euler line of $\triangle ABC$ at $P \ne O$. The circumcircle of $\triangle ACO$ meets segment $BC$ at $E \ne C$. Let $\overline{BC}$ and $\overline{FP}$ intersect at $X$, with $C$ between $B$ and $X$. If $BD=13$, $EC=8$, and $CX=27$, find $DE$.
\end{block}

\section{Problems}
Approximately increasing difficulty....
\exercise[USA TST 2004/4] Let $ABC$ be a triangle. Choose a point $D$ in its interior. Let $\omega_1$ be a circle passing through $B$ and $D$ and $\omega_2$ be a circle passing through $C$ and $D$ so that the other point of intersection of the two circles lies on $AD$. Let $\omega_1$ and $\omega_2$ intersect side $BC$ at $E$ and $F$, respectively. Denote by $X$ the intersection of $DF$, $AB$ and $Y$ the intersection of $DE, AC$. Show that $XY \parallel BC$.
\exercise[CJMO 2021/1]Let $ABC$ be an acute triangle, and let the feet of the altitudes from $A$, $B$, $C$ to $\overline{BC}$, $\ol{CA}$, $\ol{AB}$ be $D$, $E$, $F$, respectively. Points $X\neq F$ and $Y\neq E$ lie on lines $CF$ and $BE$ respectively such that $\angle XAD = \angle DAB$ and $\angle YAD = \angle DAC$. Prove that $X$, $D$, $Y$
are collinear.
\exercise[IGO 2018/I5]Suppose that $ABCD$ is a parallelogram such that $\angle DAC = 90^o$. Let $H$ be the foot of perpendicular from $A$ to $DC$, also let $P$ be a point along the line $AC$ such that the line $PD$ is tangent to the circumcircle of the triangle $ABD$. Prove that $\angle PBA = \angle DBH$. 
\exercise[Serbia 2017/6] Let $k$ be the circumcircle of $\triangle ABC$ and let $k_a$ be A-excircle .Let the two common tangents of $k,k_a$ cut $BC$ in $P,Q$.Prove that $\angle PAB=\angle CAQ$.
\exercise[Taiwan TST 2014/3/3]Let $ABC$ be a triangle with circumcircle $\Gamma$ and let $M$ be an arbitrary point on $\Gamma$. Suppose the tangents from $M$ to the incircle of 
$\triangle ABC$ intersect $\ol{BC}$ at two distinct points $X_1$ and $X_2$. Prove that the circumcircle of triangle $MX_1X_2$ passes through the tangency point of the 
$A$-mixtilinear incircle with $\Gamma$.
\exercise[USA TST 2018/5 (by Evan)] Let $ABCD$ be a convex cyclic quadrilateral which is not a kite, but whose diagonals are perpendicular and meet at $H$. Denote by $M$ and $N$ the midpoints of $\overline{BC}$ and $\overline{CD}$. Rays $MH$ and $NH$ meet $\overline{AD}$ and $\overline{AB}$ at $S$ and $T$, respectively. Prove that there exists a point $E$, lying outside quadrilateral $ABCD$, such that
\begin{itemize}
  \item ray $EH$ bisects both angles $\angle BES$, $\angle TED$, and
  \item $\angle BEN = \angle MED$.
\end{itemize}
\exercise[IMO 2019/2] In triangle $ABC$, point $A_1$ lies on side $BC$ and point $B_1$ lies on side $AC$. Let $P$ and $Q$ be points on segments $AA_1$ and $BB_1$, respectively, such that $PQ$ is parallel to $AB$. Let $P_1$ be a point on line $PB_1$, such that $B_1$ lies strictly between $P$ and $P_1$, and $\angle PP_1C=\angle BAC$. Similarly, let $Q_1$ be the point on line $QA_1$, such that $A_1$ lies strictly between $Q$ and $Q_1$, and $\angle CQ_1Q=\angle CBA$. \\[4pt]
Prove that points $P,Q,P_1$, and $Q_1$ are concyclic. 
\exercise[MOP HW \#21] In acute scalene $\triangle ABC$ with circumcenter $O$, orthocenter $H$, Kosnita point $X_{54}=K$, define $P=(HO)\cap(BOC)$, $Q$ be the foot from line onto $AO$. Prove that $P,Q,K$ are collinear. (The Kosnita point is the point at which the line through $A$ and the circumcenter of $\triangle BOC$ and the other two analogous lines concur; it is the isogonal conjugate of the nine-point center. 
\exercise[Shortlist 2012/G8]Let $ABC$ be a triangle with circumcircle $\omega$ and $\ell$ a line without common points with $\omega$. Denote by $P$ the foot of the perpendicular from the center of $\omega$ to $\ell$. The side-lines $BC,CA,AB$ intersect $\ell$ at the points $X,Y,Z$ different from $P$. Prove that the circumcircles of the triangles $AXP$, $BYP$ and $CZP$ have a common point different from $P$ or are mutually tangent at $P$.
\exercise[Shortlist 2021/G8]Let $ABC$ be a triangle with circumcircle $\omega$ and let $\Omega_A$ be the $A$-excircle. Let $X$ and $Y$ be the intersection points of $\omega$ and $\Omega_A$. Let $P$ and $Q$ be the projections of $A$ onto the tangent lines to $\Omega_A$ at $X$ and $Y$ respectively. The tangent line at $P$ to the circumcircle of the triangle $APX$ intersects the tangent line at $Q$ to the circumcircle of the triangle $AQY$ at a point $R$. Prove that $\overline{AR} \perp \overline{BC}$.\\

\subsection{Addendum}
\exercise[USAMO 2012/5] Let $P$ be a point in the plane of $\triangle ABC$, and $\gamma$ a line passing through $P$. Let $A', B', C'$ be the points where the reflections of lines $PA, PB, PC$ with respect to $\gamma$ intersect lines $BC, AC, AB$ respectively. Prove that $A', B', C'$ are collinear.
\exercise[Shortlist 2022/G8] Let $AA'BCC'B'$ be a convex cyclic hexagon such that $AC$ is tangent to the incircle of the triangle $A'B'C'$, and $A'C'$ is tangent to the incircle of the triangle $ABC$. Let the lines $AB$ and $A'B'$ meet at $X$ and let the lines $BC$ and $B'C'$ meet at $Y$. \\[3pt]
Prove that if $XBYB'$ is a convex quadrilateral, then it has an incircle.
\exercise[China 2020/2] Let ABC be a triangle, and let the bisector of $\angle A$ intersect $\ol{BC}$ at $D$. Point $P$ lies on line $AD$ such that $P, A, D$ are collinear in that order. Suppose $\ol{PQ}$ is tangent to $(ABD)$ at $Q$, $\ol{PR}$ is tangent to $(ACD)$ at $R$, and $Q$ and $R$ lie on opposite sides of line $AD$. Let $K = BR \cap CQ$. Prove that if the line through $K$ parallel to $BC$ intersects lines $QD, AD, RD$ at $E, L, F$ , respectively, then $EL = KF$ .

\subsubsection{Are these (D)DIT?}
I have not done them, but there are apparently (D)DIT solutions to the below problems.
\exercise[Shortlist 2022/G3] Let $ABCD$ be a cyclic quadrilateral. Assume that the points $Q, A, B, P$ are collinear in this order, in such a way that the line $AC$ is tangent to the circle $ADQ$, and the line $BD$ is tangent to the circle $BCP$. Let $M$ and $N$ be the midpoints of segments $BC$ and $AD$, respectively. Prove that the following three lines are concurrent: line $CD$, the tangent of circle $ANQ$ at point $A$, and the tangent to circle $BMP$ at point $B$.
\exercise[TSTST 2023/6] Let $ABC$ be a scalene triangle and let $P$ and $Q$ be two distinct points in its interior. Suppose that the angle bisectors of $\angle PAQ$,$\angle PBQ,$ and $\angle PCQ$ are the altitudes of triangle $ABC$. Prove that the midpoint of $\overline{PQ}$ lies on the Euler line of $ABC$. 
(the author was splashed again :skull:)
\end{document}